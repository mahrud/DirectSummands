\documentclass[12pt]{article}

\let\oldb\b
\usepackage[dvipsnames]{xcolor}
\usepackage{notes,hyperref}
\usepackage{mathtools}
\usepackage{cleveref}
\crefformat{equation}{(#2#1#3)} % omit eq.
\let\b\beta

%algorithms
\usepackage{algorithm}
\usepackage{algpseudocode}
\algdef{SE}[SUBALG]{Indent}{EndIndent}{}{\algorithmicend\ }%
\algtext*{Indent}
\algtext*{EndIndent}
\renewcommand{\algorithmicrequire}{\textbf{Input:}} % Changing "Require" to "Input"
\renewcommand{\algorithmicensure}{\textbf{Output:}} % Changing "Ensure" to "Output"

%\crefname{algocf}{alg.}{algs.}
%\Crefname{algocf}{Algorithm}{Algorithms}

\let\wtilde\widetilde
\let\bar\overline
\def\sing{{\mathrm{sing}}}

\def\cF{\mathcal F}
\def\OO{\mathcal O}
\def\FF{\mathbb F}
\def\PP{\mathbb P}
\def\ZZ{\mathbb Z}
\makeop{Gr}
\makeop{Bl}
\makeop{Eff}
\def\wpsi{\widehat\psi}

\let\inc\hookrightarrow
\let\xinc\xhookrightarrow

\let\d\partial

\theoremstyle{theorem}
\newtheorem{thm}{Theorem}
\numberwithin{thm}{section}
\newtheorem{algm}[thm]{Algorithm}
\newtheorem{lem}[thm]{Lemma}
\newtheorem{cor}[thm]{Corollary}
\newtheorem{prop}[thm]{Proposition}
\newtheorem{conj}[thm]{Conjecture}
\theoremstyle{definition}
\newtheorem{dfn}[thm]{Definition}
\newtheorem{exa}[thm]{Example}
\newtheorem{note}[thm]{Note}
\newtheorem{nota}[thm]{Notation}
\newtheorem{quest}[thm]{Question}
\newtheorem{rem}[thm]{Remark}
\def\defn#1{{\it #1}}

\newcommand{\mahrud}[1]{{\color{ForestGreen} \sf $\blacklozenge$ Mahrud: [#1]}}
\newcommand{\devlin}[1]{{\color{red} \sf $\clubsuit$ Devlin: [#1]}}
\def\n{\mathfrak n}

\title{Computing Direct Sum Decompositions}

\author{Devlin Mallory and Mahrud Sayrafi}

%% \author{Devlin Mallory}
%% \address{Basque Center for Applied Mathematics, Bilbao, Basque Country, Spain}
%% \email{\href{mailto:dmallory@bcamath.org}{dmallory@bcamath.org}}

%% \author{Mahrud Sayrafi}
%% \address{Max Planck Institute for Mathematics in the Sciences, Leipzig, Germany}
%% \email{\href{mailto:mahrud@mis.mpg.de}{mahrud@mis.mpg.de}}

%% \subjclass[2020]{16D70; 14F06, 13A35}

\begin{document}
\maketitle

\begin{abstract}
  We describe and prove correctness of two practical algorithms for finding indecomposable summands of finitely generated modules over a finitely generated $k$-algebra $R$. The first algorithm applies in the (multi)graded case, which enables the computation of indecomposable summands of coherent sheaves on subvarieties of toric varieties (in particular, for varieties embedded in projective space); the second algorithm applies when $R$ is local and $k$ is a finite field, which opens the door to computing decompositions in singularity theory. We also present multiple examples, including some which present previously unknown phenomena regarding the behavior of summands of Frobenius pushforwards (including in the non-graded case) and syzygies over Artinian rings.
\end{abstract}

%%%%%%%%%%%%%%%%%%%%%%%%%%%%%%%%%%%%%%%%%%%%%%%%%%%%%%%%%%%%%%%%%%%%%%%%%%%%%%%%
\section{Introduction}

The problems of finding isomorphism classes of indecomposable modules with a given property, or determining the indecomposable summands of a module, are ubiquitous in commutative algebra and representation theory and even in areas such as topological data analysis. Within commutative algebra, for instance, the classification of rings $R$ for which there are only finitely many isomorphism classes of indecomposable maximal Cohen--Macaulay $R$-modules (the \emph{finite CM-type} property), or determining whether iterated Frobenius pushforwards of a Noetherian ring in positive characteristic have finitely many isomorphism classes of indecomposable summands (the \emph{finite F-representation type} property) are two well-established research problems (see, for example, \cite{Yoshino90,LW12} for the former and \cite{SVdB97,Hara15,TT08}, among many others, for the latter). For both these problems, and many others, making and testing conjectures depends on computing summands of modules and verifying their indecomposability.
The need for and absence of an algorithm to do so is noted, for example, in
\cite[15.10.9]{Eisenbud95}.

Currently there are no algorithms available for checking indecomposability or finding summands of arbitrary modules over commutative rings. In contrast, variants of the ``Meat-Axe'' algorithm for determining irreducibility of finite-dimensional modules over a group algebra have wide ranging applications in computational group theory \cite{Parker84,HR94,Holt98} and are available through symbolic algebra software such as Magma and GAP \cite{MAGMA,GAP}.

Meanwhile, effective computation of indecomposable components of multiparameter persistence modules is essential in topological data analysis, which has been applied to computational chemistry, materials science, neuroscience, and many other areas \cite{BL23}.

The purpose of this paper is to describe and prove correctness of two practical algorithms for computing indecomposable summands of finitely generated modules over a finitely generated $k$-algebra $R$, when either $R$ is (multi)graded and $k$ an arbitrary field, or $R$ local and $k$ a field of positive characteristic.
In the graded case, our algorithm is often capable of completely decomposing a module in a single pass, while in the local case it proceeds by splitting the module into two summands and continuing recursively.
%We also describe applications of our algorithm for computations in characteristic 0.
The multigraded case of our algorithm enables the computation of indecomposable summands of coherent \emph{sheaves} on subvarieties of toric varieties (in particular, for varieties embedded in $\PP^n$).

Finally, we present multiple examples, including  previously unknown phenomena regarding the behavior of summands of Frobenius pushforwards and syzygies over Artinian rings. In particular, we highlight the results of \cite{CDE24}, which shows a recurrence formula for indecomposable summands of high syzygies of the residue field of Golod rings, made possible through experiments and observations using our algorithm.

An implementation in Macaulay2 \cite{M2} is available via the GitHub repository \\
\centerline{
  \href{https://github.com/mahrud/DirectSummands}
       {\texttt{https://github.com/mahrud/DirectSummands}}.}

%%%%%%%%%%%%%%%%%%%%%%%%%%%%%%%%%%%%%%%%%%%%%%%%%%%%%%%%%%%%%%%%%%%%%%%%%%%%%%%%

\begin{rem}
  There is a ``missing'' case of our algorithm: when $R$ is only local, not graded, and $k$ has characteristic 0. If a module over such a ring is decomposable, its reductions modulo $p$ will be as well; thus, our algorithm at least provides a heuristic for verifying decomposability in characteristic 0. See \Cref{rem:char}.
  %Although the local algorithm described below is only proved to result (probabilistically) in a decomposition into indecomposable summands in positive characteristic, in practice it often does produce nontrivial indecomposable decompositions even in characteristic 0.
  %Although the algorithm described below is only proved to result (probabilistically) in a decomposition into indecomposable summands in positive characteristic, in practice it often does produce nontrivial indecomposable decompositions even in characteristic 0. Moreover, if a module over a ring of characteristic 0 is decomposable, its reductions modulo $p$ will be as well; thus, our algorithm provides a heuristic for verifying decomposability in characteristic 0.

  We note also that while the discussion below, and our implementation in Macaulay2, concerns the case where $R$ is a \emph{commutative} ring, the basic techniques extend beyond this case. We plan to extend the results and algorithms to non-commutative rings, such as Weyl algebras, in future work.
\end{rem}

%%%%%%%%%%%%%%%%%%%%%%%%%%%%%%%%%%%%%%%%%%%%%%%%%%%%%%%%%%%%
\subsection*{Acknowledgements}

The authors would like to thank David Eisenbud for several very useful conversations and ideas, and Ezra Miller for pointing out the relevance of this algorithm in topological data analysis.
We also thank the American Institute of Mathematics for hosting the ``Macaulay2: expanded functionality and improved efficiency'' workshop, during which some of this work took place.
The work of the first author was supported in by the National Science Foundation RTG Grant No.~1840190.
The second author was supported in part by the Doctoral Dissertation Fellowship at the University of Minnesota and the National Science Foundation under Grant No.~2001101.
We also thank the referees for their careful reading and suggestions, which significantly improved the exposition of the paper.

%%%%%%%%%%%%%%%%%%%%%%%%%%%%%%%%%%%%%%%%%%%%%%%%%%%%%%%%%%%%%%%%%%%%%%%%%%%%%%%%
\section{Notation}

Throughout we will work over a field $k$ and write $\overline k$ for a choice of algebraic closure of $k$. \linebreak In particular, $\overline{\FF_p}$ is the algebraic closure of the finite field $\ZZ/p\ZZ$.

For a graded or local ring $R$ with maximal ideal $\m$ and an $R$-module $M$, we write $\mu(M):=\dim_{R/\m}(M/\m M)$ for the minimal number of generators of $M$ as an $R$-module.
If $R$ is graded and $M$ is homogeneous, we write $[M]_d$ for the $k$-vector space of degree-$d$ elements of $M$.

% TODO: \mahrud{k(x) and k(X) are used only in one place, move there?}
For an algebraic variety $X$ and a point $x\in X$, we write $k(x)=\OO_{X,x}/\m_x$ for the residue field of $X$ at $x$ and $k(X)$ for the function field of $X$.
%
When $X$ is a projective variety and $\OO_X(1)$ is a fixed very ample line bundle on $X$, the twisted global sections functor $\Gamma_*$ is the functor given by $\cF \mapsto \bigoplus_{n\in \Z} H^0(X, \cF(n))$, sending coherent sheaves to modules over the graded ring $S = \Gamma_*(\OO_X)$. Conversely, if $M$ is a finitely generated, graded $S$-module, we write $\wtilde M$ for the coherent sheaf associated to $M$.

%%%%%%%%%%%%%%%%%%%%%%%%%%%%%%%%%%%%%%%%%%%%%%%%%%%%%%%%%%%%%%%%%%%%%%%%%%%%%%%%
\section{The main algorithms}

We begin by describing the main algorithms for finding the indecomposable summands of a finitely generated module over a graded ring or a local ring with residue field contained in $\overline{\FF_p}$.

%%%%%%%%%%%%%%%%%%%%%%%%%%%%%%%%%%%%%%%%%%%%%%%%%%%%%%%%%%%%
\subsection{The graded case}\label{sec:graded-alg}

Let $R$ be a $\N$-graded ring with $R_0 = k$ a field. Suppose $M = \bigoplus M_d$ is a finitely generated, graded $R$-module and consider a general degree-zero endomorphism $\phi \in [\End M]_0$. Fix a set of minimal homogeneous generators $m_1,\dots,m_{\mu(M)}$ of $M$ so that $\phi$ may be presented as a $\mu(M)\times\mu(M)$ matrix with entries in $R$. We will refer to $\chi_\phi(\lambda) = \det(\phi-\lambda\id_M)$ as the ``characteristic polynomial of $\phi$'', which is a univariate polynomial with coefficients in $R$ and thus a priori has solutions only in the algebraic closure of the total ring of fractions of $R$.

\begin{prop}\label{prop:eigenvalues}
  Let $A\colon M/\m M\to M/\m M$ be the $k$-linear map induced by $\phi\in[\End M]_0$. The solutions of the characteristic polynomial $\chi_\phi(\lambda)$ are the same as the eigenvalues of $A$, thus they lie in the algebraic closure $\bar k$.
\end{prop}
\begin{proof}
  Since $M$ is $\N$-graded, assume its generators are ordered with $\deg m_i$ nonincreasing, and let $d_1,\dots,d_n$ be the decreasing list of unique degrees of generators of $M$. The matrix for $\phi$ is then ``block upper triangular'' with $n$ diagonal blocks with entries in $k$, i.e., of the form
  \begin{equation}\label{eq:blocks}
    \phi = \begin{pmatrix}
    B_1    & *      & \dots  & *      \\
    0      & B_2    & \dots  & *      \\
    \vdots & \vdots & \ddots & \vdots \\
    0      & 0      & \dots  & B_n
    \end{pmatrix},
  \end{equation}
  where the blocks $B_i$ are square matrices \emph{with entries in $k$}, and the remaining entries are in $\m$. To see this, note that since $\phi$ has degree zero, $\phi(m_i)$ is a $k$-linear combination of the $m_j$ with $\deg m_j = \deg m_i$ plus an $R$-linear combination of the $m_j$ with $\deg m_j < \deg m_i$. %In particular, $B_i$ is an $k$-endomorphism on the $k$-vector space spanned by all generators of degree $d_i$.

  Then the characteristic polynomial $\det(\phi - \lambda \id_M) = \prod \det(B_i - \lambda \id) = \det(\phi - \lambda \id_{M/\m M})$ is a univariate polynomial in $\lambda$ with coefficients in $k$, thus has solutions in $\bar k$.
\end{proof}

We will refer to the collection of $k$-eigenvalues of $B_i$ as the ``eigenvalues of $\phi$''. Further, let $\mu_j \coloneqq \mu(\lambda_j)$ be the sum of the multiplicities of $\lambda_j$ in each $B_i$. Note that the multiplicity of $\lambda_j$ in $\chi_\phi(\lambda)$ is the sum of the multiplicities of $\lambda_j$ in each $\chi_{B_i}(\lambda)$.

\begin{rem}\label{rem:grading}
  The hypotheses that $R$ is $\N$-graded may be weakened. In the proof of \Cref{prop:eigenvalues}, we only need a partial order on the degrees of the generators of $M$ in order to present $\phi$ in a block upper triangular form \eqref{eq:blocks}. Suppose $R$ is $\ZZ^r$-graded, then the effective cone $\Eff(R)\subset\Q^n$ generated by the elements $\deg m$ for each monomial $m \in R$ induces a partial order on the degrees of generators of $M$ if it is strongly convex (contains no positive dimensional subspace of $\Q^n$) and $R_0 = k$. We will call such rings \emph{positively graded}. % TODO: can R_0 be finite dimensional?
  This is satisfied, for instance, in the setting of multiparameter persistence modules or when $R$ is the Cox ring of a toric variety or Mori dream space graded by the class group of the variety.
\end{rem}

We will use nonzero eigenvalues of $\phi$ to get nontrivial splittings of $M$. As the following lemma shows, this approach allows for decomposing multiple summands at once.

\begin{lem}\label{lem:decomposition}
  Suppose $\psi_1,\dots,\psi_r$ are endomorphisms of $M$ such that $M\to\im\psi_i$ is a split surjection for all $i$ and set $\psi = \psi_r\circ\dots\circ\psi_1$.
  %If $\ker\psi_i\cap\ker(\psi_{i-1}\circ\dots\circ\psi_1) = 0$ for all $i$ then $M$ has a direct sum decomposition \( M = \ker\psi_r \oplus \cdots \oplus \ker\psi_1 \oplus \im\psi. \)
  If $\ker\psi_i\cap\ker \psi_j = 0$ for all $i\neq j$ then $M$ has a direct sum decomposition \( M = \ker\psi_r \oplus \cdots \oplus \ker\psi_1 \oplus \im\psi. \)
\end{lem}
\begin{proof}
  Since each $\psi_i$ is a split surjection, there are direct sum decompositions $M=\im \psi_i \oplus \ker \psi_i$, and thus idempotents $e_i$ with $\im e_i = \im \psi_i$ and $\ker e_i = \ker \psi_i$.

  The $e_i$ must commute: let $f_i = 1- e_i$ be the complementary idempotent (projection onto $\ker \psi_i$). Since $\ker \psi_i \cap \ker \psi_j=0$, $f_if_j=0$. In particular,
  \[ e_ie_j =(1-f_i)(1-f_j) = 1-f_i-f_j= 1-f_j-f_i = e_j e_i. \]

  Since the $e_i$ are commuting idempotents, $e_r\cdots e_1$ is also an idempotent. It's clear that $\im(e_r\cdots e_1) = \im (\psi_r\circ \dots\circ \psi_1) = \im \psi$. Moreover, we claim that $\ker(e_r\cdots e_1) = \ker \psi_r \oplus \cdots \oplus \ker \psi_1$. To see this, note that since $f_if_j=0$ we get
  \[ 1-e_r\cdots e_1 = 1-(1-f_r)\cdots (1-f_1) = \sum f_i. \]
  In other words, the kernel of $e_r\cdots e_1$ is exactly $\bigoplus \ker e_i =\bigoplus \ker \psi_i$.

  Thus, the idempotent $e_r\cdots e_1$ corresponds to a direct sum decomposition
  \[ M = \ker(e_r\cdots e_1) \oplus \im(e_r\cdots e_1) = \ker\psi_r \oplus \cdots \oplus \ker\psi_1 \oplus \im\psi, \]
  as desired.
\end{proof}

Note that as long as each $\psi_i$ in \Cref{lem:decomposition} is nontrivial (not zero or identity endomorphisms), the kernel summands will be nontrivial, however, they may not necessarily be indecomposable. Also note that $\im\psi$ may be zero.

\pagebreak
\begin{prop}\label{prop:split-surj}
  Let $\lambda_1,\dots,\lambda_r$ be the eigenvalues of $\phi$ over $k$ and set $\psi_j = (\phi - \lambda_j\id_M)^{\mu_j}$ and $\psi = \psi_1\circ\cdots\circ\psi_r$. Then $M$ has a direct sum decomposition
  \( M = \ker\psi_1 \oplus \cdots \oplus \ker\psi_r \oplus \im\psi. \)
\end{prop}
\begin{proof}
  First, suppose $\lambda_{r+1},\dots,\lambda_t$ are the other eigenvalues of $\phi$ over the algebraic closure $\bar k$ and set $\rho = \psi_{r+1}\circ\dots\circ\psi_t$, then by the Cayley--Hamilton theorem $\chi_\phi(\phi) = \psi_1\circ\dots\circ\psi_r\circ\rho = \psi\circ\rho = 0$. Since the factors in $\chi_\phi(\phi)$ can be reordered, we can write $0 = \psi\circ\rho = \rho \circ \psi$. In particular, suppose the desired direct sum decomposition existed over $\bar k$, then we have \mahrud{WHY??}
  \begin{align*}
    M = \ker\psi_1\oplus\dots\oplus\ker\psi_r\oplus(\ker\psi_{r+1}\oplus\dots\oplus\ker\psi_t)
    &= \ker\psi_1\oplus\dots\oplus\ker\psi_r\oplus\ker\rho \\
    &= \ker\psi_1\oplus\dots\oplus\ker\psi_r\oplus\im\psi.
  \end{align*}
  Thus it suffices to prove the direct sum decomposition only over the algebraic closure.
  
  If $r = 1$ then $\im\psi_1=0$ is zero, hence we get the trivial decomposition $M = M \oplus 0$.

  %  We will find a right inverse to $\psi_1$ and $\psi_2$ using $\chi_\phi(\phi) = 0$
  If $r = 2$, let $A\colon M/\m M\to M/\m M$ be the $k$-linear map induced by $\phi$ and $A_i\coloneqq(A - \lambda_i)^{\mu_i}$ the map induced by $\psi_i$ for $i=1,2$. Observe that
  \begin{align*}
  \chi_\phi(\phi) = (\phi - \lambda_1\id_M)^{\mu_1} (\phi - \lambda_2\id_M)^{\mu_2}
  &= \left(\phi^{\mu_1} - \dots + c_1\phi - c_0\id_M\right)\circ\psi_2 \\
  &= \left(\phi^{\mu_1} - \dots + c_1\phi\right)\circ\psi_2 - c_0\psi_2 = 0,
  \end{align*}
  where $c_0 = -(-\lambda_1)^{\mu_1}$ and $c_1 = (-\lambda_1)^{\mu_1-1}$, and therefore we have a left pseudoinverse
  \[
  \psi_2
  = c_0^{-1} \left(\phi^{\mu_2} - \dots + c_1\phi \right) \circ\psi_2
  % = \phi\circ c_0^{-1} \left(\phi^{\mu_2-1} - \dots + c_1 \right) \circ\psi_2.
  \]
  This means we have a short exact sequence $0 \to \im\psi_2 \hookrightarrow M \to \coker\psi_2 \to 0$ which is split by the map $c_0^{-1} \left(\phi^{\mu_2} - \dots + c_1\phi \right)$, % which is to say $M \to \im\psi_2$ is a split surjection.

  \begin{align*}
  \chi_\phi(\phi) = (\phi - \lambda_2\id_M)^{\mu_2} (\phi - \lambda_1\id_M)^{\mu_1}
  &= \psi_2\circ\left(\phi^{\mu_1} - \dots + c_1\phi - c_0\id_M\right) \\
  &= \psi_2 c_0 - \psi_2\circ \left(\phi^{\mu_1} - \dots + c_1\phi\right) = 0,
  \end{align*}



  
  % $\lambda_1^{-1}A_1$ is an idempotent and by \Cref{lem:idemp} (and scaling by $\lambda_1$ for simplicity) we get $M = \ker\psi_1 \oplus \im\psi_1$.
% \devlin{But $\lambda_1\inv A_1$ is not guaranteed to be an idempotent: e.g., if $\lambda_1 = 1$ with multiplicity 2, then $A_1$ has the block $\begin{pmatrix}1 & 2 \\ 0 & 1\end{pmatrix}$ (i.e., a $2\times 2$ Jordan block squared), and it's not an idempotent.}

  Otherwise, observe that $\wpsi_j\coloneqq \psi_1\circ\dots\circ\psi_{j-1}\circ\psi_{j+1}\circ\dots\circ\psi_r $ has only one nonzero eigenvalue over $k$, namely $\widehat\lambda_j \coloneqq \prod_{i\neq j}(\lambda_j-\lambda_i)$: Using \Cref{prop:eigenvalues}, eigenvalues of $\phi$ are the same as the eigenvalues of $A$. Say $A v = \lambda_\ell v$ for $v\in M/\m M$ and apply the $k$-linear map induced by $\psi_1\circ\dots\circ\psi_r$ to get
  \begin{align}\label{eq:eigenvalues-psi}
    \prod(A-\lambda_i)v
    &= (A - \lambda_1)\cdots(A - \lambda_{r-1})(\lambda_\ell - \lambda_r)v \nonumber \\
    &= (A - \lambda_1)\cdots(\lambda_\ell - \lambda_{r-1})(\lambda_\ell - \lambda_r)v
    = \dots = \prod(\lambda_\ell - \lambda_i)v,
  \end{align}
  where the product is over $i=1,\dots,r$. In particular, since the $\lambda_i$ are distict the $k$-linear map $\widehat A_j\coloneqq (A - \lambda_1)\dots(A-\lambda_{j-1})(A - \lambda_{j+1})\dots(A - \lambda_r)$ induced by $\wpsi_j$ removes the $(\lambda_\ell-\lambda_j)$ factor from \eqref{eq:eigenvalues-psi}, hence the eigenvalue of $\widehat A_j$ corresponding to $v$ is $\prod_{i\neq j}(\lambda_j-\lambda_i)$ if $\lambda_\ell=\lambda_j$ and 0 otherwise, with multiplicities $\mu_j$ and $\mu(M) - \mu_j$ respectively.

  In particular, $\sfrac{\widehat A_j}{\widehat\lambda_j}$ is an idempotent, so by \Cref{lem:idemp} (and scaling by $\widehat\lambda_j$ for simplicity) we have $M = \ker\wpsi_j \oplus \im\wpsi_j$ and further each $\psi_j$ is a split surjection $M\to\im\psi_j = \ker\wpsi_j$. Finally, note that elements of $\ker\psi_i = \im\wpsi_i$ must be generated by columns of $\phi$ corresponding to the eigenvalue $\lambda_i$, hence $\ker\psi_i \cap \ker\psi_j = 0$ for $i\neq j$, thus by \Cref{lem:decomposition} we have the desired decomposition.
\end{proof}

%% \begin{proof}
%%   First we show that each $\psi_j$ is a split surjection $M\to\im\psi_j$, hence $M = \ker\psi_j \oplus \im\psi_j$. To see this, note that $R$-linear row and column operations on endomorphisms of $M$ correspond to pre- and post-composition by $R$-linear maps on $M$. In particular, performing row and column operations only involving elements of $R_0 = k$ on $\phi$ we may transform each $B_i$ into Jordan canonical form
%%   \begin{equation}
%%     J_i = \begin{pmatrix}
%%       J_{i,1} & 0       & \dots  & 0      \\
%%       0       & J_{i,2} & \dots  & 0      \\
%%       \vdots  & \vdots  & \ddots & \vdots \\
%%       0       & 0       & \dots  & J_{i,t},
%%     \end{pmatrix}
%%   \end{equation}
%%   where each Jordan block has entries over $k$ and a fixed eigenvalue. Suppose $J_{i,j}$ is the block with eigenvalue $\lambda_j$, then $N_i = J_{i,j} - \lambda_j$ is a nilpotent matrix whose size is the multiplicity of $\lambda-\lambda_j$ in $\chi_{B_i}(\lambda)$. Since $\mu_j$ is the sum of the multiplicities of $\lambda_j$ in each $B_i$, the matrix $N_i^{\mu_j}$ vanishes, hence $(J_i-\lambda_j)^{\mu_j}$ is a matrix with some 1s along the diagonal and 0s elsewhere. This implies that there are auto morphisms $\alpha$ and $\beta$ of $M$ (corresponding to column operations clearing the entries above and to the right of each nonzero diagonal) such that $\alpha \circ (\phi - \lambda_j\id_M)^{\mu_j} \circ \beta: M\to M$ is represented by a diagonal matrix $\psi'_j$ with 1s and 0s along the diagonal, which is a split surjection.

%%   Finally, if $\lambda_j$ and $\lambda_k$ are different then the nilpotent blocks in $J_i-\lambda_j$ and $J_i-\lambda_k$ are distinct, therefore $\ker\psi_j \cap \ker\psi_k = 0$, 
%% \end{proof}

We are ready to state the algorithm for decomposition of a graded module. Recall from \Cref{rem:grading} that we only need the ring $R$ to be positively graded over an arbitrary field.

\begin{algorithm}[H]
  \caption{(Indecomposable summands of a graded module over a commutative ring)}\label{alg:graded}
  \begin{algorithmic}[1]
    \smallskip
    \Require graded module $M$ over a positively graded commutative $k$-algebra $R$.
    \Ensure  list $\textproc{Summands}(M)$ of indecomposable summands of $M$.
    \State Initialize a list $L$.
    \State Take a general element $\phi$ in $[\End_R(M)]_0$,
           the degree zero part of $\End_R(M)$. \label{item:End0}
    \ForAll { eigenvalues $\lambda_i$ of $\phi$ }
      \State Let $\mu_i$ be the multiplicity of $\lambda_i$.
      \State Let $\psi_i = (\phi - \lambda_i \id_M)^{\mu_i}$.
      \State Append \Call{Summands}{$\ker\psi_i$} to $L$.
    \EndFor
    \State Let $\psi$ be the composition of all $\psi_i$ above.
    \If {$\psi$ is nonzero}
      \State Append \Call{Summands}{$\im\psi$} to $L$.
    \EndIf
    \State \Return $L$.
  \end{algorithmic}
\end{algorithm}

We will see via \Cref{lem:distinct} that this will result in an indecomposable decomposition of $M$ over any field.

\begin{rem}
Note that \Cref{alg:graded} does not depend on the characteristic of the field.
Since over a large enough field (i.e., when $|k|$ is larger than the number of summands) the eigenvalues of a general endomorphism $\phi$ are distinct (by a variant of the argument of \Cref{lem:term}),
the summands $\ker\psi_i$ and $\im\psi$ will be indecomposable; thus \Cref{alg:graded} will produce indecomposable summands in one non-recursive iteration. 
When $|k|$ is less than or approximately the number of summands, the algorithm will be recursive.
\end{rem}

%%%%%%%%%%%%%%%%%%%%%%%%%%%%%%%%%%%%%%%%%%%%%%%%%%%%%%%%%%%%
\subsection{The local case}\label{sec:local-alg}

Throughout this section, $(R,\m)$ will be either local with maximal ideal $\m$, or a (multi)graded ring with $R_0 = k \subset \overline{\FF_p}$ a field of positive characteristic and homogeneous maximal ideal $\m$.

We begin with the observation that if $M$ is a finitely generated $R$-module and $A\in\End_R(M)$ is an idempotent, then $M$ decomposes as $\im A \oplus \coker A$. If $A$ is neither an isomorphism nor the zero morphism, then both factors are nonzero and $M$ is decomposable.

Note that $A$ also acts on the $k$-vector space $M/\m M$.

The following lemma allows us to check only for idempotents modulo the maximal ideal.

\begin{lem}\label{lem:idemp}
  Let $M$ be a finitely generated $R$-module, and let $A\in\End_R(M)$. If the induced action of $A$ on $M/\m M$ is idempotent, then $M$ admits a direct sum decomposition $\im A \oplus \coker A$.
\end{lem}
\begin{proof}
  By assumption, we can write $A^2 = A + B$, where $B\in\End_R(M)$ with $B(M)\subset \m M$.
  Note that if $x \in \m^kM$, then $A^2(x) - A(x) = B(x)$ lies in $\m^{k+1}M$.

  Let $N = \im A$. We want to show that $0 \to N\inc M$ splits. Consider the composition
  \[ N\inc M \xra{A} \im A = N.\]
  We claim that this composition is surjective. Since a surjective endomorphism of finitely generated modules is invertible, we would then conclude that this composition is an isomorphism on $N$; say $\a$.
  Therefore the inclusion \( 0 \to N \inc M \) is split by the map of $R$-modules $M\xra{A} N\xra{\a\inv} N$, and thus $M$ decomposes as claimed.

  To check the surjectivity of $N\inc M \xra{A} N$,  we may complete at the maximal ideal and thus assume $R$, $M$, and $N$ are complete. Let $n_0\in N$. By assumption, $n_0=A(m_1)$ for some $m_1\in M$. Applying $A$ again, we get
  \[ A(n_0) = A^2(m_1) = A(m_1) + n_1 = n_0 + n_1, \]
  or equivalently
  \[ n_0 = A(n_0) - n_1 \]
  for some $n_1\in \m M$. In fact, since $n_0$ and $A(n_0)$ are both in $N$, we have $n_1\in N$ also, so $n_1\in \m M\cap N$.

  Thus, we can write $n_1 = A(m_2)$ for $m_2\in M$.
  Now, apply $A$ to both sides: by the assumption that $A$ is idempotent modulo $\m$, we have
  \[ A(n_1)=A^2(m_2) = A(m_2) + n_2 = n_1+n_2, \]
  Thus, $n_2=A(n_1)-n_1$, so $n_2\in \m^2M$; clearly also $n_2\in N$ as well.
  Combining the previous equations, we can write
  \[ n_0 = A(n_0) - n_1 = A(n_0) - A(n_1) + n_2 = A(n_0 - n_1) + n_2, \]
  with $n_1\in \m M\cap N$ and $n_2\in \m^2M\cap N$.

  Continuing in this fashion, for any $k$ we can write
  \[ n_0=A(n_0-n_1+\dots \pm n_k) \mp n_{k+1}, \]
  with $n_i \in \m^i M\cap N$.

  By the Artin--Rees lemma \cite[Lemma~5.1]{Eisenbud95}, there's some positive integer $k$ such that for $n\gg0$ we can write
  \[ \m^n M\cap N = \m^{n-k} ( \m^kM\cap N)\subset \m^{n-k} N. \]
  That is, the terms of $n_0-n_1+n_2-\cdots$ are going to 0 in the $\m$-adic topology on $N$. Hence we can write
  \[ n_0=A(n_0-n_1+n_2-\cdots), \]
  with $n_0-n_1+n_2-\cdots\in N$. We conclude that $A$ is surjective as a map $N\to N$.
\end{proof}

Thus, if we produce an element $A\in \End_R(M)$ that is an idempotent modulo $\m$, we obtain a splitting of $M$.
The following lemma allows us to produce idempotents modulo $\m$.

\begin{lem}\label{lem:jordan}
  Let $k$ be a finite field of characteristic $p$, and let $A$ be an endomorphism of a $k$-vector space such that all eigenvalues of $A$ are contained in $k$. If $\lambda$ is an eigenvalue of $A$, then some power of $A-\lambda I$ is idempotent.
  Furthermore, if $\lambda$ is not the only eigenvalue of $A$, then a  nonzero power of $A-\lambda I$ is idempotent.
\end{lem}
\begin{proof}
  Since all eigenvalues of $A$ are contained in $k$, we can without loss of generality put $A$ in Jordan canonical form, with each Jordan block having the form
  \[ r_i\Biggl\{
  \underbrace{\begin{pmatrix}
      \lambda_i & 1 & 0 & \dots & 0 \\
      0 & \lambda_i & 1  & \dots & 0 \\
      \vdots & & & & \vdots \\
      0 & 0 & 0 & \dots & \lambda_i
  \end{pmatrix}}_{r_i} \]
  with each $\lambda_i$ an eigenvalue of $A$.
  In this basis, $A-\lambda I$ will be block-diagonal with blocks of form
  \[ \begin{pmatrix}
    \lambda_i-\lambda & 1 & 0 & \dots & 0 \\
    0 & \lambda_i-\lambda & 1 & \dots & 0 \\
    \vdots & & & & \vdots \\
    0 & 0 & 0 & \dots & \lambda_i-\lambda
  \end{pmatrix}. \]
  Set $\mu_i=\lambda_i-\lambda$. Then for any $l\geq 1$, the $l$-th power $(A-\lambda I)^l$ is block-diagonal with blocks of the form
  \[ \begin{pmatrix}
    \mu_i^l & \binom{l}1\mu_i^{l-1} & \binom{l}2\mu_i^{l-2} & \dots & \binom{l}{r_i}\mu_i^{l-r_i} \\
    0 & \mu_i^l & \binom{l}1\mu_i^{l-1} & \dots & \binom{l}{r_i-1}\mu_i^{l-r_i+1} \\
    \vdots & & & & \vdots \\
    0 & 0 & 0 & \dots & \mu_i^l
  \end{pmatrix}. \]
  If we choose $l > r_i$ and $l$ a power of $p$, say $p^{e_0}$, then all non-diagonal terms will vanish, so all blocks will have the form
  \[ \begin{pmatrix}
    \mu_i^l & 0 & 0 & \dots & 0 \\
    0 & \mu_i^l & 0 & \dots & 0 \\
    \vdots & & & & \vdots \\
    0 & 0 & 0 & \dots & \mu_i^l
  \end{pmatrix}. \]
  Finally, since $k\subset \FF_{p^e}$ for some $e$, if we choose $l$ to be divisible also by $p^e-1$, we have that $$\mu_i^{l-1} = (\mu_i^{p^{e}-1})^{l/(p^{e}-1)} =1.$$
  In particular, taking $l=p^{e_0}(p^e-1)$,
  $(A-\lambda I)^{p^{e_0}(p^e-1)}$ is a diagonal matrix with diagonal entries 1 or 0, hence idempotent. Note moreover that if some $\lambda_i\neq \lambda$, then $(A-\lambda I)^l$ is not the zero matrix.
\end{proof}


This leads to a probabilistic algorithm to find the indecomposable summands of a finitely generated $R$-module $M$ in our setting, as follows:

\begin{algorithm}[H]
  \caption{(Indecomposable summands of a module over a commutative local ring)}\label{alg:local}
  \begin{algorithmic}[1]
    \smallskip
    % TODO: finitely generated M? noetherian R?
    \Require module $M$ over a commutative local ring $R$.
    \Ensure  list $\textproc{Summands}(M)$ of indecomposable summands of $M$.
    \State Initialize a list $L$.
    \State Take a general element $\phi$ in $\End_R(M) \setminus \m \End_R(M)$.
    \State Let $\phi_0$ be the induced endomorphism of the $k$-vector space $M/\m M$.
    % TODO: Why did we need to assume $A$ has at least two eigenvalues?
    \State Let $\lambda$ be any eigenvalue of $\phi_0$.
    \State Let $\psi = (\phi - \lambda \id_M)^{p^{e_0} (p^{e} - 1)}$
    (with the power explicitly as in the proof of \Cref{lem:jordan}).
    \State By \Cref{lem:idemp}, $\psi$ is an idempotent, hence producing a splitting of $M$.
    \If { both $\coker\psi$ and $\im\psi$ are nonzero }
      \State Append \Call{Summands}{$\coker\psi$} to $L$.
      \State Append \Call{Summands}{$\im\psi$} to $L$.
    \Else
      \State Append $M$ to $L$.
    \EndIf
    \State \Return $L$.
  \end{algorithmic}
\end{algorithm}

%%%%%%%%%%%%%%%%%%%%%%%%%%%%%%%%%%%%%%%%%%%%%%%%%%%%%%%%%%%%
\subsection{Termination of the algorithms}

As a consequence of \Cref{alg:graded} and \Cref{alg:local} above, we have:

\begin{lem}\label{lem:term}
  Assume that either $(R,\m)$ is a graded ring with $R_0=k$, or $(R,\m)$ is a local ring with $k=R/\m$ a field of positive characteristic, and that $M$ is a finitely generated $R$-module (homogeneous in the graded case).
  If $M$ has an endomorphism with at least two distinct eigenvalues modulo $\m$, then $M$ is not indecomposable.
\end{lem}

The converse observation implies that if $M$ is not indecomposable, then these algorithms will find the direct sum decomposition of $M$:

\begin{lem}\label{lem:distinct}
  Retain the notation of \Cref{lem:term}.
  \begin{itemize}
  \item If $R$ is graded and $M$ is not indecomposable, then a general endomorphism of $M$ of degree zero has at least two distinct eigenvalues (in $\bar k$)
  \item If $R$ is local and $M$ is not indecomposable, then a general endomorphism of $M$ of degree zero reduces to an endomorphism of $M/\m M$ with at least two distinct eigenvalues.
  \end{itemize}
\end{lem}

\begin{rem}
  By ``general'' we mean that a general linear combination of a basis for $[\End_R(M)]_0$ (in the graded case) or of minimal generators of $\End_R(M)$ (in the local case) over the algebraic closure of $k$, or equivalently over a sufficiently large algebraic extension of~$k$.
\end{rem}



In particular, if $M$ is not indecomposable, then \Cref{alg:graded} and \Cref{alg:local} produce a nontrivial direct sum decomposition of $M$, because a general element of $\End M$ will have distinct eigenvalues and thus the algorithms will produce a nontrivial decomposition.

\begin{proof}
  We may assume that the base field $k$ is algebraically closed.
  Let $\Phi_1,\dots,\Phi_r$ be a basis for $[\End_R(M)]_0$ in the graded case, or minimal generators of $\End_R(M)$ in the local case, and $\phi_1,\dots,\phi_r$ their images modulo $\m$, which we view as matrices with entries in $k$.
  Note that if $M$ is decomposable, say $M=M_1\oplus M_2$, then we may always take $\Phi_1,\Phi_2$ to be projectors onto each summand.
  Let $U\subset \A^r$ be the subset of $r$-tuples $(\lambda_1,\dots,\lambda_r)$ such that $\lambda_1\phi_1+\dots+\lambda_r\phi_r$ has at least two distinct eigenvalues, i.e., such that $\lambda_1\Phi_1+\dots+\lambda_r\Phi_r$ reduces to an endomorphism of $M/\m M$ with at least two distinct eigenvalues. (Note that in the graded case the eigenvalues of the reduction of an endomorphism modulo $\m$ are the same as the eigenvalues of endomorphism, so there is no harm in looking only at $M/\m M$.)

  It suffices to show that $U$ is a nonempty open subset of $\A^r$. First, we show $U$ is nonempty:
  Say $M=M_1\oplus M_2$ with $M_1,M_2$  nonzero  summands. Choose $\Phi_1$ to be the projection to $M_1$, and $\Phi_2$ the projection to $M_2$. Then for any $\lambda_1,\lambda_2\in k$, $\lambda_1\phi_1 + \lambda_2\phi_2$ has eigenvalues $\lambda_1,\lambda_2$; thus in particular there is \emph{an} element of $\End_R(M)$ reducing to an endomorphism of $M/\m M$ with distinct eigenvalues, so $U$ is nonempty.

  Now, we show that $U$ is open. This is a purely linear algebraic statement: we claim that given a matrix $\phi$ with at least two distinct eigenvalues, and any $r$ matrices $\phi_1,\dots,\phi_r$, that
  $$ A_{\lambda_1,\dots,\lambda_r}:=\phi+\lambda_1\phi_1+\dots+\lambda_r\phi_r $$
  has at least two distinct eigenvalues for $\lambda_1,\dots,\lambda_r$ outside a Zariski-closed subset of $\A^r$.
  The eigenvalues of $A_{\lambda_1,\dots,\lambda_r}$ are the roots of $\det(A_{\lambda_1,\dots,\lambda_r}-t I)$, which is a polynomial in $t$ with coefficients in $\lambda_1,\dots,\lambda_r$.
  $A_{\lambda_1,\dots,\lambda_r}$ fails to have at least two distinct eigenvalues exactly when this polynomial factors as a power of a linear term.

  This condition is polynomial in the coefficients of powers of $t$ in $\det(A_{\lambda_1,\dots,\lambda_r}-t I)$ and thus in the $\lambda_i$; to see this, note that
  if we write
  $$ f:=\det(A_{\lambda_1,\dots,\lambda_r}-t I)=t^n  +t^{n-1}b_{n-1}+\dots +t b_1 +b_0, $$
  then $f$
  has an $n$-fold root exactly when
  $ f,\d f/\d t,\dots, \d^{n-1} f/\d t^{n-1} $
  vanish simultaneously; the resultant of these $n$ polynomials in the $n$ variables $b_i$ gives polynomial conditions in the $b_i$ for this to occur.
  In our setting, the $b_i$ are themselves polynomials in the $\lambda_i$, and thus we have obtained polynomial equations defining the locus where $A_{\lambda_1,\dots,\lambda_r}$ fails to have distinct roots, and thus the complement $U$ is open.
\end{proof}


\begin{rem}
  Note that 
\Cref{alg:graded} and \Cref{alg:local} are 
 quite sensitive to the ground field $k$, because they needs all the eigenvalues of the endomorphism 
%$A$ of $M/\m M$ 
to be contained in $k$. While theoretically the issue can be avoided by working over an algebraically closed ground field $\bar k$, for practical use on a computer algebra system it is better to extend $k$ to some larger finite field. However, the general linear combinations we take in Step~\ref{item:End0} of \Cref{alg:graded} should be taken with respect to the prime subfield (otherwise, as we increase the size of the finite field $k$, the eigenvalues of a general linear combination will live in higher and higher field extensions).
  See \Cref{ex:elliptic} for a demonstration of the necessity of extending the base field.

We also note that the algorithm detects the necessity of a field extension: if the characteristic polynomial of the endomorphism does not factor over $k$, the algorithm will calculate the splitting field of $k$; the user can then opt to extend the ground field to this splitting field.
\end{rem}

If the above algorithm fails to produce a nontrivial idempotent, it does not certify that $M$ is indecomposable. However, there are a few sufficient conditions to be indecomposable, which in practice often (but not always) produce such a certification.
The following sufficiency condition is immediate, but can be quite useful in practice for verifying indecomposability:

\begin{lem}
  Suppose $R$ is graded, and let $M$ be a finitely generated $R$-module and let $[\End_R(M)]_0$ be the $k$-vector space of degree zero endomorphisms. Suppose that either:
  \begin{enumerate}
  \item $[\End_R(M)]_0$ is 1-dimensional and thus spanned by the identity endomorphism, or
  \item every non-identity element of $[\End_R(M)]_0$, viewed as a matrix, has entries in $\m$;
  \end{enumerate}
  then $M$ is indecomposable.
\end{lem}

\begin{proof}
  If $M$ decomposes non-trivially as $M_1\oplus M_2$, then the projections onto each factor are nontrivial degree zero endomorphisms not equal to the identity, and which do not have entries contained in~$\m$.
\end{proof}

The following lemma is the algebraic analogue of the following fact about coherent sheaves:

\begin{cor}
  Let $X$ be a projective variety over a field $k$, and $\cF$ a coherent sheaf on $X$.
  If $ H^0(\End \cF) = k, $ then $\cF$ is indecomposable.
\end{cor}

\begin{proof}
  If there is a nontrivial direct sum decomposition $\cF = \cF_1\oplus \cF_2$, then the composing the projection and inclusions we obtain $\cF\to \cF_1\inc \cF$ yields a nonzero endomorphism of $\cF$ that is a scalar multiple of $\id_\cF$, and thus $H^0(\End \cF) \neq k$.
\end{proof}

\begin{rem}\label{rem:char}
  If $R$ is a local ring over a field $k\subset \C$ of characteristic 0, and $M$ a finitely generated $R$-module, it would be useful to have an algorithm to find a direct sum decomposition of $M$.
  At the moment, we do not have such an algorithm outside the graded case.
  However, we note that the algorithm above can be used to test indecomposability of $M$ via reduction modulo $p$, as follows:
  One can choose a finitely generated $\Z$-algebra $A$ and an $A$-algebra $R_A$ such that $R_A\otimes _A k=R$, and likewise an $R_A$-module $M_A$ such that $M_A\otimes _A k = R$ and such that $M_A$ is flat over $A$. If $\n$ is a maximal ideal of $A$, one can check that $A/\n \cong \FF_{p^e}$ for some prime $p$ and $e$.
  The key point is that if $M$ is decomposable, then $M_A$ also can be taken to be decomposable. Thus, the various reductions  $M\otimes A/\n$ will also be decomposable for all $\n$.
  If our algorithm does not detect an indecomposable summand of $M\otimes A/\n$ for an appropriate choice of $\n$, then, the original module $M$ must have been indecomposable.
  However, if $M$ is \emph{decomposable}, we do not have a way of patching the decompositions of various $M_A\otimes_A A/\n$ into a decomposition of $M$.

  We also point out that it is possible to ``guess'' an idempotent for $M$, even when there is no algorithm to produce one. In practice, if $M$ does have nontrivial idempotents, Macaulay2 often chooses them as some of the generators of $\End_R(M)$. By checking if the generators of $\End_R(M)$ are idempotent, and trying to construct idempotents by subtracting eigenvalues off of general endomorphisms and checking idempotency, we can often produce nontrivial direct sum decompositions in characteristic 0.
\end{rem}

\begin{rem}
  We omit a computational complexity analysis here because without any assumptions about the module there are too many parameters that in any given application may become prominent. We note, however, that $[\End_R(M)]_0$ can be computed using degree-limited syzygy algorithms without computing all of $\End_R(M)$. While in many applications this significantly reduces the computational complexity, in general the complexity of the algorithm presented here is at least bounded by the complexity of computing syzygies. Compare with \cite{DX22}, which provides a complexity analysis for a decomposition algorithm with the additional assumption that the generators of $R$ as well as generators and relations of $M$ have distinct degrees. This assumption is exceedingly rare in commutative algebra and algebraic geometry applications, but it is relevant, though not guaranteed, in the context of multiparameter persistence in topological data analysis.
\end{rem}

%%%%%%%%%%%%%%%%%%%%%%%%%%%%%%%%%%%%%%%%%%%%%%%%%%%%%%%%%%%%%%%%%%%%%%%%%%%%%%%%
\section{Decomposing coherent sheaves}\label{sec:coherent}

While the preceding section was written in the language of modules, by the standard translation to global (multi)projective varieties, the algorithm works equally well to find indecomposable decompositions of coherent sheaves on (multi)projective varieties.
In this section, we make a few notes in the single-graded case regarding the relation between the eigenvalues discussed in the preceding section (in this section, called \emph{module-theoretic eigenvalues}) with the notion of eigenvalue of an endomorphism of a vector bundle.

Let $X\subset \P^n$ be a projective variety, with ample line bundle $\OO_X(1)$.
Throughout, let $E$ be a vector bundle (i.e., a locally free coherent sheaf) on $X$.

The following is well-known:

\begin{lem}{{\cite[Exercise~4.1]{Friedman98}}}
  A morphism $f:E\to E$ is injective if and only if it is an isomorphism.
\end{lem}

%(It is also true that $f$ is surjective if and only if it's an isomorphism, but this is more well-known.)

%For completeness, we give the standard proof (see, e.g., \cite[Exercise~4.1]{Friedman}).

%\begin{proof}
%This requires only that $E$ is a coherent sheaf:
%% since $\End_{\OO_X}(E)$ is finite-dimensional, $f$ satisfies some minimal monic polynomial $\phi(x)$; since $f$ is injective, the constant term of $\phi(x)$ is nonzero.
%We claim that $f\otimes k(x)$ is injective for any $k(x)$. This then implies that $f\otimes k(x)$ is surjective for every $x\in X$ and thus $f$ is surjective.
%To see this, note that $f$ satisfies some minimal-degree monic polynomial, since $\End_{\OO_X}(E)$ is finite-dimensional over $k$, and that this monic polynomial has nonzero constant term, since $f$ is injective. Now, $f\otimes k(x)$ satisfies this same polynomial, so must be injective.
%\end{proof}

\begin{dfn}
  For $f\in \End_{\OO_X}(E)$, taking top exterior powers yields the map
  $$ \bigwedge^{\rank E} f: \det E  \to \det E, $$
  where $\det E := \bigwedge^{\rank E} E$.
  Since $\End_{\OO_X}(\det E) = H^0(\OO_X)=k$, $\bigwedge^{\rank E} f$ is multiplication by $\lambda\in k$; we write $\det f$ for this scalar $\lambda$.
\end{dfn}

\begin{lem}
  Let $x \in X$ be any point (not necessarily closed), with residue field $k(x) = \OO_{X,x}/\m_x$. Then $\det(f\otimes k(x)) =\det f$.
  %In other words, if we restrict $f$ to the fiber $E_x$, the determinant is the same as $\det f$.
\end{lem}

\begin{proof}
  Since $(\bigwedge^{\rank E} f)\otimes k(x) = \bigwedge^{\rank( E\otimes k(x))}( f\otimes k(x))$ as $k(x)$-linear maps on $E\otimes k(x)$,
  we have $\det(f\otimes k(x)) =\det(f)\otimes k(x)$, but $\det(f)\in k$ and so is unaffected by going modulo $\m_x$.
\end{proof}



\begin{lem}
  $\det f \neq 0 $ if and only if $f$ is injective (if and only $f$ is an isomorphism).
\end{lem}

\begin{proof}
  Let $\ker f \neq 0$. Since localization is exact
  we have
  $\ker(f\otimes k(X)) = (\ker f)\otimes k(X)$.
  Since $\ker f \subset E$ is torsionfree, the localization map $\ker f \to (\ker f)\otimes k(X)$ is injective and thus $(\ker f) \otimes k(X)\neq 0$.
  Since the $k(X)$-vector space map $f\otimes k(X)$ is not injective, $\det(f\otimes k(X))= 0$.
  In particular  $\det f = 0$ as well.
  Conversely,
  if $\ker f = 0$, then
  $f$ is an isomorphism, hence an isomorphism on fibers, and hence
  $\det(f\otimes k(x)) \neq 0$ for any $x$. Thus $\det f\neq 0$.
\end{proof}

\begin{dfn}
  $\lambda \in k$ is an \defn{eigenvalue} of $f \in \End_{\OO_X}(E)$ if $\det(f-\lambda \id_E)= 0$. In other words, the eigenvalues of $f$ are the zeroes of the univariate polynomial $\det(f-t\id _E)$ with coefficients in $k$.
\end{dfn}

Note that $f$ is an isomorphism if and only if $\lambda=0$ is not an eigenvalue of $f$, just as for ordinary linear operators on vector spaces.

\begin{dfn}
  Let $M$ be a homogeneous module over a graded ring $R$ with $R_0=k$.
  Let $g \in \End_R(M)$. The \defn{module-theoretic eigenvalues} of $g$ are defined to be the eigenvalues of the map of vector spaces $g\otimes R/\m:M/\m M \to M/\m M$.
\end{dfn}

In other words, the module-theoretic eigenvalues are the eigenvalues modulo the maximal ideal discussed in the previous section.


Now, let $R$ be the homogeneous coordinate ring of $X\subset \P^n$, and let $M$ be a graded $R$-module such that $\wtilde M=E$.

\begin{lem}
  Let $f:E\to E$ arise from a map $g:M\to M$ (i.e., $\wtilde M = E$ and $\wtilde g = f$).
  The eigenvalues of $f$ are a subset of the module-theoretic eigenvalues of $g$.
\end{lem}

Note that any endomorphism of $E$ corresponds to an endomorphism of $\Gamma_*(E)$; we may want the freedom to work with other module representatives of $E$ though.

\begin{proof}
  All that needs to be shown is that if $\det(f-\lambda \id_E) =0$, then $\det((g-\lambda \id_M)\otimes R/\m)=0$.
  Replacing
  $f-\lambda \id_E$
  by
  $f$
  and likewise
  $g-\lambda\id_M$
  by
  $g$,
  we just need to show that $\det(f)=0$ implies $\det(g\otimes R/\m)=0$.

  Say $\det(f)=0$.
  If $\det(g\otimes R/\m)\neq 0$, then $g$ induces a surjection $M/\m M\to M/\m M$, and thus $g$ induces a surjection $M\to M$ by Nakayama's lemma. The endomorphism $f$ is thus an isomorphism \cite[Corollary~4.4]{Eisenbud95}, and so $\det(f)\neq 0$.
\end{proof}

\begin{exa}
  By adding  irrelevant summands to $M$, one can always add extraneous module-theoretic eigenvalues, so the containment of eigenvalues may be proper.
  %However, if $M=\Gamma_*(E)$, set of values will be the same: in this case, if $g\otimes R/m$ is not an isomorphism, then $g$ cannot be surjective.
  Note that in general the multiplicities will never be equal, even if one takes $M=\Gamma_*(E)$: if $f$ is multiplication by $\lambda$, then $f$ will have $\rank E$ many eigenvalues, while $g$ will have $\mu(M)$ module-theoretic eigenvalues.
\end{exa}

By combining the discussion above with the results of \Cref{sec:graded-alg}, we thus have:

\begin{prop}
  If $X$ is a variety over a field $k$,
  $E$ a vector bundle on $X$, and $f:E\to E$ has two distinct eigenvalues $\lambda_1,\lambda_2$, then $E$ is a nontrivial direct sum, with one summand given by $(f-\lambda_1)^{N}$ for some $N$.
\end{prop}

\begin{proof}
  Take $M= \Gamma_*(E)$; then since the eigenvalues $\lambda_1\neq \lambda_2$ are also module-theoretic eigenvalues of $M$, \Cref{alg:graded} produces a nontrivial direct sum decomposition $M_1\oplus M_2$, and since $M$ is torsion-free we know that $\widetilde M_i\neq 0$, and thus $E=\widetilde M_1\oplus \widetilde M_2$ is a nontrivial direct summand of sheaves.
\end{proof}

\begin{rem}
  We note that it is known already by \cite[Proposition~15]{Atiyah57} that an endomorphism of an indecomposable vector bundle $E$ cannot have two distinct eigenvalues. The utility of the preceding lemma is in producing an explicit direct sum decomposition of $E$.
\end{rem}


%%%%%%%%%%%%%%%%%%%%%%%%%%%%%%%%%%%%%%%%%%%%%%%%%%%%%%%%%%%%%%%%%%%%%%%%%%%%%%%%
\section{Examples}\label{sec:examples}

In this section, we give examples of the kind of calculations and observations \Cref{alg:graded} and \Cref{alg:local} allow us to make.

\renewcommand{\char}{\operatorname{char}}

\begin{exa}[Frobenius pushforward on the projective space $\P^n$]
  Let $S = k[x_0,\dots,x_n]$ be a polynomial ring with $\char k = p$ and $\deg x_i = 1$ and consider the Frobenius endomorphism
  \[ F\colon S\to S \quad \text{given by} \quad f \to f^p. \]
  Hartshorne \cite{Hartshorne1970} proved that for any line bundle $L\in\Pic\P^n$, the Frobenius pushforward $F_*L$ splits as a sum of line bundles. While the following calculations are straightforward to do by hand, they are immediately calculated via our algorithm:
  \begin{align*}
    \text{When }p=3, n=2: \\
    F_*\OO_{\P^2} &= \OO \oplus \OO(-1)^7 \oplus \OO(-2). \\
    \text{When } p=2, n=5: \\
    F_*\OO_{\P^5} &= \OO \oplus \OO(-1)^{15} \oplus \OO(-2)^{15} \oplus \OO(-3), \\
    F_*^2\OO_{\P^5} &= \OO \oplus \OO(-1)^{120} \oplus \OO(-2)^{546} \oplus \OO(-3)^{336} \oplus \OO(-4)^{21}.
  \end{align*}
\end{exa}

\begin{exa}[Frobenius pushforward on toric varieties]
  Let $X$ be a smooth toric variety and consider its Cox ring
  \[ S = \bigoplus_{[D]\in\Pic{X}} \; \Gamma(X, \OO(D)). \]
  Similar to the case of the projective space, B{\o}gvad and Thomsen \cite{Bogvad98,Thomsen00} showed that $F_*L$ totally splits as a direct sum of line bundles for any line bundle $L\in\Pic X$.

  As an example, consider the third Hirzebruch surface $X=\P(\OO_{\P^1}\oplus \OO_{\P^1}(3))$ over a field of characteristic 3. We have, for example, that
  \begin{align*}
    F_*\OO_X      &= \OO_X   \oplus \OO_X(-1,0)^2 \oplus \OO_X(0,-1)^2 \oplus \OO_X(1,-1)^3 \oplus \OO_X(2,-1), \\
    F_*\OO_X(1,1) &= \OO_X^3 \oplus \OO_X(-1,0)   \oplus \OO_X(1,-1)   \oplus \OO_X(1, 0)^2 \oplus \OO_X(2,-1)^2.
  \end{align*}
  In fact, Achinger \cite{Achinger15} showed that the total splitting of $F_*L$ for every line bundle $L$ characterizes smooth projective toric varieties.
  % The toric Frobenius is simply multiplication by $p$ on the torus.
  % The $p$ need not be prime, nor even match the characteristic!
\end{exa}

\begin{exa}[Frobenius pushforward on elliptic curves]\label{ex:elliptic}
  Consider the elliptic curve
  \[ X = \Proj \FF_7[x,y,z]/(x^3+y^3+z^3). \]
  %%Weierstrauss form is y^2=x^3-432
  This is an ordinary elliptic curve, hence $F$-split; thus $\OO_X$ is a summand of $F_* \OO_X$. Over the algebraic closure of $\FF_7$, $F_*\OO_X$ will decompose as $\bigoplus_{p=1}^7 \OO_X(p_i)$, where $p_1,\dots,p_7$ are the 7-torsion points of $X$.

  However, over $\FF_7$, our algorithm calculates that $F_*\OO_X$ decomposes only as
  $$ F_* \OO_X =\OO_X \oplus M_1\oplus M_2\oplus M_3, $$
  with $M_i$ indecomposable (over $\FF_7$) of rank 2.

  If one extends the ground field to $\FF_{49}$, however, our algorithm calculates the full decomposition
  $$ F_* \OO_X=\bigoplus_{p=1}^7 \OO_X(p_i). $$
  This reflects the fact that the 7-torsion points $p_i$ of $X$, and thus the sheaves $\OO_X(p_i)$, are not defined over $\FF_7$, but rather are defined over $\FF_{49}$.
\end{exa}

\begin{exa}[Frobenius pushforward on Grassmannians]
  % https://arxiv.org/abs/1901.10956
  Consider the Grassmannian $X = \Gr(2,4)$. We may work over the Cox ring $S$,
  %\[ S = \bigoplus\!{}_{[D]\in\Pic{X}} \; \Gamma(X, \OO(D)). \]
  which in this case coincides with the coordinate ring
  \[ S = \frac{k[p_{0,1},p_{0,2},p_{0,3},p_{1,2},p_{1,3},p_{2,3}]}{p_{1,2}p_{0,3}-p_{0,2}p_{1,3}+p_{0,1}p_{2,3}}. \]
  Then in characteristic $p=3$ we have:
  \[ F_*\OO_X = \OO \oplus \OO(-1)^{44} \oplus \OO(-2)^{20} \oplus A^4 \oplus B^4, \]
  where $A$ and $B$ are rank-2 indecomposable bundles (c.f.~\cite{RSVdB22}).
\end{exa}

\begin{exa}[Frobenius pushforward on Mori Dream Spaces]
  Continuing with the theme of computations over the Cox ring, the natural geometric setting is to consider the class of projective varieties known as Mori dream spaces \cite{HK00}.

  For instance, consider $X = \Bl_4\P^2$, the blowup of $\P^2$ at 4 general points. We work over the $\ZZ^5$-graded Cox ring
  \[ S = k[x_1,\dots,x_{10}]/\text{(five quadric Pl\"ucker relations)} \]
  with degrees
  \[
  \left(\!\begin{array}{rrrrrrrrrr}
  0&0&0&0&1&1&1&1&1&1 \\
  1&0&0&0&-1&-1&-1&0&0&0 \\
  0&1&0&0&-1&0&0&-1&-1&0 \\
  0&0&1&0&0&-1&0&-1&0&-1 \\
  0&0&0&1&0&0&-1&0&-1&-1
  \end{array}\!\right).
  \]
  Then in characteristic 2 we have:
  \begin{align*}
    F_*^2\OO_X = {\OO_{X}^{1}}
    &\oplus {\OO_{X}^{2}\ \left(-2,\,1,\,1,\,1,\,1\right)} \oplus {\OO_{X}^{2}\ \left(-1,\,0,\,0,\,0,\,1\right)} \\
    &\oplus {\OO_{X}^{2}\ \left(-1,\,0,\,0,\,1,\,0\right)} \oplus {\OO_{X}^{2}\ \left(-1,\,0,\,1,\,0,\,0\right)} \\
    &\oplus {\OO_{X}^{2}\ \left(-1,\,1,\,0,\,0,\,0\right)} \oplus B \oplus G,
  \end{align*}
  where $B, G$ are rank-3 and rank-2 indecomposable modules, as calculated in \cite{Hara15}.
\end{exa}

\begin{exa}[Frobenius pushforward on cubic surfaces]
  Let $X$ be a smooth cubic surface. Aside from a single exception in characteristic 0, $X$ will be globally $F$-split, so that any $F^e_*\OO_X $ admits $\OO_X$ as a direct summand.
  The other summands of Frobenius pushforwards of $\OO_X$ have yet to be studied, and in particular it is not known whether such rings should have the finite $F$-representation type property.

  The use of our algorithm to compute examples in small $p$ and $e$ suggest the following behavior:
  $$ F_* \OO_X = \OO_X\oplus M, $$
  with $M$ indecomposable, and furthermore $F_*^e M$ remains indecomposable for all $e\geq 0$. In other words, the indecomposable decomposition of $F^e_* \OO_X$ is
  $$ F_*^e \OO_X \cong \OO_X\oplus M\oplus F_* M\oplus\dots\oplus F_*^{e-1}M. $$
  In particular, this suggests $\OO_X$ will fail to have the finite $F$-representation type property.
  In fact, we believe a similar description holds true for quartic del Pezzos.
\end{exa}

%\begin{exa}[Frobenius pushforward on invariant rings of finite groups]
% https://arxiv.org/abs/2312.11786
%  \mahrud{TODO: add an invariant ring example from \cite{HS24}}
%\end{exa}

\begin{exa}[Local singularities]
  Let $R=\F_2[x,y,z]_{(x,y,z)}/(x^2y+xy^2+xyz+z^2)$. In the notation of \cite{Artin77}, this is the $D_4^1$ singularity, which is an exceptional version of the usual rational double point/Du Val singularities appearing in characteristic 2.  In particular, note that $R$ is not homogeneous. If $F_* R$ denotes the Frobenius pushforward of $R$, then the algorithm of \Cref{sec:local-alg} computes the following indecomposable summands:
  $$
  F_*R = R\oplus
  \coker
  \begin{pmatrix}
    x+y+z&z\\
    z&xy
  \end{pmatrix}\oplus\coker \begin{pmatrix}
    y&z\\
    z&x^{2}+xy+xz
  \end{pmatrix}\oplus \coker\begin{pmatrix}
  x&z\\
  z&xy+y^{2}+yz
  \end{pmatrix};
  $$
  there is one free summand and three reflexive modules of rank 1.
\end{exa}


\begin{exa}[Syzygies over Artinian rings]
  % https://arxiv.org/abs/2208.05427
  In recent work suggested by examples calculated using our algorithm, \cite{CDE24} studied the indecomposable summands of syzygy modules over a Golod ring $(R,\m,k)$ and found previously unexpected recurring behavior. Specifically, the syzygy modules of the residue field are direct sums of only three indecomposable modules: the residue field $k$, the maximal ideal $\m$, and an additional module $N = \Hom_R(\m, R)$.

  Here, we give a concrete example of one such ring. Let $k$ be any field and let $R = k[x,y]/(x^3,x^2y^3,y^5)$ and consider the (infinite) minimal free resolution of the residue field, which has rank $2^n$ in homological index $n$.   The fourth syzygy module of the residue field decomposes (ignoring the grading) as the direct sum
  $$ k^3 \oplus \m^2 \oplus N^3, $$
  and the fifth syzygy module as
  $$ k^8\oplus \m^9 \oplus N^2, $$
  where the module $N$ can be explicitly presented as
  \[ N = \coker
  \begin{pmatrix}
    x^{2}&0&0&0&y^{4}&xy^{3}&0&0\\
    -y&x&y^{3}&0&0&0&0&0\\
    0&0&0&y&-x&0&0&0\\
    0&0&0&0&0&-y&x&y^{2}
  \end{pmatrix} \]

  The use of our algorithm was essential to the observation that beyond the ``guaranteed'' summands of $k$ and $\m$ (which were known to appear by work of \cite{DE23}) only the one additional indecomposable module $N$ appears in the summands of syzygies of $k$.
  %% needsPackage "DirectSummands"
  %% R = ZZ/101[x,y]/(ideal"x3,x2y3,y5")
  %% ideal(x^2*y^2,x*y^3)
  %% F = res (cokernel vars R, LengthLimit => 6)
  %% netList apply(length F, i -> summands coker F.dd_i)
\end{exa}

%\begin{exa}[]
%  \mahrud{TODO: example about Auslander-Ritten quivers}
%\end{exa}

\begin{exa}[Symbolic diagonalization]\label{ex:diagonalization}
  An interesting application of our algorithm, suggested by Bernd Sturmfels,  is automated diagonalization of symbolically parameterized matrices. As a simple demonstration, let $R = K[a,b,c,d]$ and consider the following matrix
  \[ A = \begin{pmatrix}
    a&b&c&d\\
    d&a&b&c\\
    c&d&a&b\\
    b&c&d&a
  \end{pmatrix} \]
  Then the splittings of $\coker A$ over $K = \Q$ and $K = \Q(i)$ have the following presentations, respectively:
  \[\medmuskip0mu \begin{pmatrix}
    a+b+c+d&0&0&0\\
    0&a-b+c-d&0&0\\
    0&0&a-c&b-d\\
    0&0&b-d&c-a
  \end{pmatrix},
  \,
  \begin{pmatrix}
    a+b+c+d&0&0&0\\
    0&a-b+c-d&0&0\\
    0&0&a+i\,b-c-i\,d&0\\
    0&0&0&a-i\,b-c+i\,d
  \end{pmatrix}.
  \]
\end{exa}

%\begin{exa}[Multiparameter persistence modules]
%  \mahrud{TODO: add an invariant ring example from \cite{BL23} or \cite{DX22}}
%\end{exa}

%%%%%%%%%%%%%%%%%%%%%%%%%%%%%%%%%%%%%%%%%%%%%%%%%%%%%%%%%%%%%%%%%%%%%%%%%%%%%%%%

\bibliographystyle{alpha}
\bibliography{references}
\end{document}


% Local Variables:
% tab-width: 2
% eval: (visual-line-mode)
% End:
