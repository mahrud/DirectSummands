\documentclass[beamer]{paper}

\usepackage{quiver}
\usepackage[italic,symbolgreek]{mathastext}

% for highlighting
\usepackage{xcolor, soul}
\makeatletter
\let\HL\hl
\renewcommand\hl{%
  \let\set@color\beamerorig@set@color
  \let\reset@color\beamerorig@reset@color
  \HL}
\makeatother

\usetheme{Warsaw}
\beamertemplatenavigationsymbolsempty
%\setbeamertemplate{footline}[frame number]
%\setbeamercovered{transparent} % gray out hidden stuff
\AtBeginSection{\frame{\sectionpage}}

\newcommand{\lc}{\MakeLowercase}

\newcommand{\PPN}{\PP{\underline{\nn}}}

\newcommand{\reg}{\operatorname{reg}}

\newcommand{\cH}{\mathcal{H}} % Hirzebruch surface

\renewcommand{\aa}{\mathbf a}
\newcommand{\bb}{\mathbf b}
\newcommand{\cc}{\mathbf c}
\newcommand{\dd}{\mathbf d}
\newcommand{\ee}{\mathbf e}
\newcommand{\vv}{\mathbf v}
\newcommand{\ww}{\mathbf w}
\newcommand{\pp}{\mathbf p}
\newcommand{\qq}{\mathbf q}
\newcommand{\nn}{\mathbf n}
\newcommand{\one}{\mathbf 1}
\newcommand{\zero}{\mathbf 0}
\newcommand{\Om}{\Omega}
\newcommand{\om}{\omega}
\newcommand{\cech}{\check{C}}
\newcommand{\Alt}{\bigwedge\nolimits}
\newcommand{\R}{\mathbf R} % right derived

\def\OO{\mathcal{O}}
\def\FF{\mathbb{F}}

\def\a{\alpha}
\def\b{\beta}
\def\g{\gamma}
\def\del{\partial}
\def\cech{\check{C}}
\def\Gs{\Gamma_{\!*}}

\def\ub{\underbrace}
\def\on{\operatorname}
\def\ph{\phantom{-}}

\def\Sym{\operatorname{Sym}}

\def\L{\mathcal{L}}
\def\E{\mathcal{E}}
\def\X{\mathcal{X}}
\def\B{\mathbf{B}}
\def\RR{\mathbf{R}}
\def\cR{\mathcal{R}}
\def\L{\mathbf{L}}
\def\U{\mathbf{U}}
\def\H{\mathbb{F}}
\def\F{\mathcal{F}}
\def\G{\mathcal{G}}
\def\C{\mathcal{C}}
\def\Z{\ZZ}

\def\Db{\operatorname{D}^{\operatorname{b}}} % -> \Db{X}
\def\coh{\operatorname{coh}}
\def\cL{\mathcal{L}}

\def\from{\leftarrow}
\def\into{\hookrightarrow}
\def\operatornameto{\twoheadrightarrow}

\def\id{\operatorname{id}}
\def\th{\operatorname{th}}
\def\colim{\operatorname{colim}}
\def\lenef{\le_{\operatorname{nef}}}
\def\lnef{<_{\operatorname{nef}}}
\def\Dsat{D_{\operatorname{sat}}}

\def\cS{\mathcal{S}}
\def\cT{\mathcal{T}}
\def\cK{\mathcal{K}}
\def\cB{\mathcal{B}}
\def\tS{\widetilde{S}}
\def\cU{\mathcal{U}}
\def\cW{\mathcal{W}}
\def\cC{\mathcal{C}}
\def\tr{\widetilde{\rho}}
\def\ts{\widetilde{\sigma}}
\def\tt{\widetilde{\tau}}

\newcommand{\PG}[1]{{\color{PineGreen}#1} }
\newcommand{\dao}[1]{{\color{DarkOrchid}#1} }

\begin{document}

% Use Computer Modern with Serif Roman Font Family
\rmfamily
\setbeamerfont{frametitle}{family=\rmfamily\selectfont}

\title[Computing summands of the Frobenius over Cox rings]
      {\textsc{Computing Summands of the Frobenius \\ over Multigraded Cox Rings}}

% I acknowledge that the University of Minnesota stands on Miní Sóta Makhóčhe, the homelands of the Dakhóta Oyáte.

% Abstract: It is known that the Frobenius pushforward of the structure sheaf on toric varieties splits as a sum of line bundles, and high enough pushforwards of it generate the derived category. Working over the multigraded Cox ring of a Mori Dream Space, we discuss algorithms for determining the summands of the Frobenius pushforward of the structure sheaf.

\author[Mahrud Sayrafi]{M\lc{ahrud} S\lc{ayrafi} \\
  {\footnotesize U\lc{niversity} of M\lc{innesota}, T\lc{win} C\lc{ities}} \\[0em]
  {\footnotesize Joint work with Devlin Mallory (University of Utah)}}

%\institute[UMN]{U\lc{niversity} of M\lc{innesota}, T\lc{win} C\lc{ities}}

% https://www.jointmathematicsmeetings.org/meetings/national/jmm2024/2300_progfull.html
\date{{\footnotesize AMS Special Session on \\
    Derived Categories, Arithmetic, and Geometry, \\
    Joint Mathematics Meetings} \\ January 4, 2024}

\frame[plain]{\titlepage}


\begin{frame}[t]{Warm-up: Frobenius pushforward on $\PPn$}
  Let $\kk$ be an algebraically closed field of char $p>0$.

  Let $S = \kk[x_0,\dots,x_n]$ be a polynomial ring with $\deg x_i = 1$.

  \vfill
  %\pause
  Consider the Frobenius endomorphism
  \[ F\colon S\to S \quad \text{given by} \quad f \to f^p \]

  %\pause
  - Hartshorne: any $L\in\Pic\PPn$, $F_*L$ splits as a sum of line bundles. \\
  %\pause
  - Raczka '24: any v.b.\ on $\PPn$, $F_*^e\cE$ splits as a sum of $\OO$'s and $\Omega$'s.
  %\pause
  \begin{block}{Example: $F_*\o_{\PPn}$}
    \vspace*{-0.2in}
    \begin{align*}
      \text{When $p=3$, $n=2$:} \\
      F_*\o_{\PP2} = \o \oplus &\o(-1)^7 \oplus \o(-2). \\
      \text{When $p=2$, $n=5$:} \\
      F_*\o_{\PP5} = \o \oplus &\o(-1)^{15} \oplus \o(-2)^{15} \oplus \o(-3). \\
      F_*^2\o_{\PP5} = \o \oplus &\o(-1)^{120} \oplus \o(-2)^{546} \oplus \o(-3)^{336} \oplus \o(-4)^{21}
    \end{align*}
  \end{block}
\end{frame}


\begin{frame}[t]{Warm-up: Frobenius pushforward on toric varieties}
  Let $X$ be a smooth toric variety and consider the Cox ring
  \[ S = \bigoplus\!{}_{[D]\in\Pic{X}} \; \Gamma(X, \OO(D)) \]

  %\pause
  \begin{theorem}[Thm.~3.2, \cite{[Cox95]}]
    Every coherent sheaf on a simplicial toric variety $X$ corresponds \\
    to a finitely generated $\Pic X$-graded $S$-module.
  \end{theorem}

  \vfill
  %\pause
  Fact: for any toric variety, $S$ is a \textbf{multigraded} polynomial ring.
  %\pause
  Upshot: we can do commutative algebra on a toric variety! \\
  %\pause
  \hfill + easy monomial orderings, Gr\"obner bases, syzygies, etc.
\end{frame}


\begin{frame}[t]{Warm-up: Frobenius pushforward on toric varieties}
  Let $X$ be a smooth toric variety and consider the Cox ring
  \[ S = \bigoplus\!{}_{[D]\in\Pic{X}} \; \Gamma(X, \OO(D)) \]

  %\pause
  - Bogvad '98, Thomsen '00: $F_*L$ totally splits for any $L\in\Pic X$.
  %\pause
  \begin{block}{Example: $X = $ Hirzebruch surface of type 3, $F_*\o_X, p = 3$}
    \vspace*{-0.2in}
    \[ F_*\o_X = \o_X \oplus \o_X(-1,0)^2 \oplus \o_X(0,-1)^2 \oplus \o_X(1,-1)^3 \oplus \o_X(2,-1). \]
  \end{block}

  - Achinger '13: \\ \hfill splitting of $F_*L$ characterizes smooth projective toric varieties.

  \vfill
  %\pause
  \begin{block}{Remark}
    The toric Frobenius is simply multiplication by $p$ on the torus.
  \end{block}
  %\pause
  \begin{alertblock}{Funky Remark}
    The $p$ need not be prime, nor even match the characteristic!
  \end{alertblock}
\end{frame}


\begin{frame}{Frobenius pushforward on Grassmannians}
  We can still consider the Cox ring
  \[ S = \bigoplus\!{}_{[D]\in\Pic{X}} \; \Gamma(X, \OO(D)) \]
  which in this case coincides with the coordinate ring.

  %\pause
  \begin{block}{Example: $X = \mathrm{Gr(2,4)}, F_*\o_X, p = 3$}
    Working over the coordinate ring:
    \[ S = \frac{\kk[p_{0,1},\dots p_{0,2},p_{1,2},p_{0,3},p_{1,3},p_{2,3}]}{p_{1,2}p_{0,3}-p_{0,2}p_{1,3}+p_{0,1}p_{2,3}} \]
    When $p = 3$:
    \[ F_*\o_X = \o \oplus \o(-1)^{44} \oplus \o(-2)^{20} \oplus A^4 \oplus B^4, \]
    where $A$ and $B$ are rank 2 indecomposable bundles. \\
    \hfill (c.f. [Raedschelders--Špenko--Van den Bergh, '17])
  \end{block}
\end{frame}


\begin{frame}{Frobenius pushforward on Mori Dream Spaces}
  This is the natural setting to work over the Cox ring
  \[ S = \bigoplus\!{}_{[D]\in\Pic{X}} \; \Gamma(X, \OO(D)). \]
  Hu--Keel: $X$ is an $MDS$ $\iff$ the Cox ring is finitely generated.

  \vfill
  %\pause
  \begin{exampleblock}{Examples}
  \; - Projective toric varieties \\
  \; - Grassmannians and flag varieties \\
  \; - Determinantal varieties \\
  \; - Many complete intersection rings \\
  \; - Smooth Fano varieties over $\CC$ \\
  \; - $\overline{M}_{0,n}$ for $n\leq 6$ (7,8,9 are unknown)
  \end{exampleblock}
\end{frame}

\begin{frame}{Frobenius pushforward on Mori Dream Spaces}
  \begin{block}{Example: $X = \mathrm{Bl}_4\PP2, F_*\o_X, p = 2$}
    Working over the $\ZZ^5$-graded Cox ring \vspace*{-0.15in}
    \[ S = \kk[x_1,\dots,x_{10}]/\text{(five quadric Pl\"ucker relations)} \]
    %\pause
    With degrees \vspace*{-0.15in}
    \[
    \left(\!\begin{array}{cccccccccc}
      0&0&0&0&1&1&1&1&1&1 \\
      1&0&0&0&-1&-1&-1&0&0&0 \\
      0&1&0&0&-1&0&0&-1&-1&0 \\
      0&0&1&0&0&-1&0&-1&0&-1 \\
      0&0&0&1&0&0&-1&0&-1&-1
    \end{array}\!\right)
    \]
  \end{block}
\end{frame}


\begin{frame}{Frobenius pushforward on Mori Dream Spaces}
  \begin{block}{Example: $X = \mathrm{Bl}_4\PP2, F_*\o_X, p = 2$}
    \vspace*{-0.15in}
    \begin{align*}
      F_*^2\o_X = {\o_{X}^{1}}
      &\oplus {\o_{X}^{2}\ \left(-2,\,1,\,1,\,1,\,1\right)} \oplus {\o_{X}^{2}\ \left(-1,\,0,\,0,\,0,\,1\right)} \\ 
      &\oplus {\o_{X}^{2}\ \left(-1,\,0,\,0,\,1,\,0\right)} \oplus {\o_{X}^{2}\ \left(-1,\,0,\,1,\,0,\,0\right)} \\
      &\oplus {\o_{X}^{2}\ \left(-1,\,1,\,0,\,0,\,0\right)} \oplus B \oplus G,
    \end{align*}
    where $B, G$ are rank 3 and rank 2 indecomposables. \\
    \hfill (c.f. [Hara, '15])
  \end{block}
\end{frame}


\begin{frame}{The splitting algorithm}
  \begin{alertblock}{The ``Meataxe'' algorithm}
    Input: finitely generated graded module $M$ over an algebra.
    %\pause
    \begin{enumerate}
    \item Compute $\Hom(M, M)$; \\ %\pause
      Note: the degree zero part only, so \textbf{multigrading helps!} %\pause
    \item Find element $h$ corresponding to an idempotent; \\ %\pause
      \begin{itemize}
      \item Hope that generators of $\Hom$ are idempotents; %\pause
      \item In finite characteristic, use a general endomorphism. %\pause
      \end{itemize}
    \item Find the image and cokernel of $h$;
    \item Repeat.
    \end{enumerate}
  \end{alertblock}

  \vfill
  %\pause
  \begin{lemma}
    Let $M$ be an $R$-module, and let $A\in\End_R(M)$. If the induced action of $A$ on $M/mM$ is idempotent, then $M$ admits a direct sum decomposition $N\oplus M/N$, where $N\cong \im A$.
    % Let $A$ be an endomorphism of a $k$-vector space such that all eigenvalues of $A$ are contained in $k$. If $\lambda$ is an eigenvalue of $A$, then some power of $A-\lambda I$ is an idempotent.
  \end{lemma}

\end{frame}


\begin{frame}[t]{Frobenius pushforward on determinantal varieties}
  \textbf{Fact:} a determinantal variety $X$ is a Mori dream space. \\
  \hfill (i.e. the Cox ring is a finitely generated algebra.)

  \vfill
  Singh: does $F^e_*\OO_X$ have finitely many classes of summands? \\
  This is known as having \emph{globally finite $F$-representation type}.

  \begin{block}{Problem}
    Classify MDS with globally finite $F$-representation type.
  \end{block}

  \begin{alertblock}{Upshot}
    Determining line bundle summands is easy over the Cox ring.
  \end{alertblock}

  \vfill
\end{frame}


\begin{frame}[t]{References}
  \scriptsize{
    - Hartshorne \hfill\textit{Ample subvarieties of algebraic varieties} \\
    - Bogvad, \hfill\textit{Splitting of the direct image of sheaves under the Frobenius} \\
    - Thomsen, \hfill\textit{Frobenius direct images of line bundles on toric varieties} \\
    - Achinger, \hfill\textit{A characterization of toric varieties in characteristic $p$} \\
    - Hara, \hfill\textit{Looking out for Frobenius summands on a blown-up surface of $\PP2$} \\
    - Raedschelders, Špenko, Van den Bergh, \hfill\textit{The Frobenius morphism in invariant theory}
  }

  \renewcommand{\section}{\subsection}
  \printbibliography
\end{frame}

\end{document}

% Local Variables:
% tab-width: 2
% eval: (visual-line-mode)
% End:
